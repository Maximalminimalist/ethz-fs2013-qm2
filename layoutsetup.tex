%% Memoir layout setup

%% NOTE: You are strongly advised not to change any of them unless you
%% know what you are doing.  These settings strongly interact in the
%% final look of the document.

% Dependencies
\usepackage{array}

% Define the default sans serif font as the lighter computer modern bright by
% D. Knuth.
\renewcommand{\sfdefault}{cmbr}

% Turn extra space before chapter headings off.
\setlength{\beforechapskip}{0pt}

% Chapter style redefinition
\makeatletter
\newcommand\thickhrulefill{\leavevmode \leaders%
\hrule height 6.25pt depth -3.25pt \hfill \kern \z@}
\setlength\midchapskip{10pt}
\makechapterstyle{VZ14}{
  \renewcommand\chapternamenum{}
  \renewcommand\printchaptername{}
  \renewcommand\chapnamefont{\Large\scshape}
  \renewcommand\printchapternum{%
    \chapnamefont\null\thickhrulefill\quad
    \@chapapp\space\thechapter\quad\thickhrulefill}
  \renewcommand\printchapternonum{%
    \par\thickhrulefill\par\vskip\midchapskip
    \hrule\vskip\midchapskip
  }
  \renewcommand\chaptitlefont{\Huge\scshape\centering}
  \renewcommand\afterchapternum{%
    \par\nobreak\vskip\midchapskip\hrule\vskip\midchapskip}
  \renewcommand\afterchaptertitle{%
    \par\vskip\midchapskip
\hrule\nobreak\vskip\afterchapskip}
}

% Set the way pages are layed out (headers and page numbering)
\if@twoside
  \pagestyle{Ruled}
\else
  \pagestyle{ruled}
\fi

% Use the newly defined style
\chapterstyle{VZ14}

% Redefine sectional headings to contain rules
\renewcommand{\section}{\@startsection{section}{1}{0mm}%
{-2\baselineskip}{0.8\baselineskip}%
{\hrule depth 0.2pt width\textwidth\hrule depth1.5pt%
width0.25\textwidth\vspace*{1.2em}\Large\bfseries\sffamily}}

\renewcommand{\subsection}{\@startsection{subsection}{2}{0mm}%
{-2\baselineskip}{0.8\baselineskip}%
{\hrule depth 0.2pt width\textwidth\hrule depth1pt width0.25\textwidth\vspace*{0.8em}\large\bfseries\sffamily}}

\renewcommand{\subsubsection}{\@startsection{subsubsection}{3}{0mm}%
{-2\baselineskip}{0.8\baselineskip}%
{\large\bfseries\sffamily}}

\setparaheadstyle{\normalsize\bfseries\sffamily}
\setsubparaheadstyle{\normalsize\bfseries\sffamily}

% Set captions to a more separated style for clearness
\captionnamefont{\sffamily\bfseries\footnotesize}
\captiontitlefont{\sffamily\footnotesize}
\setlength{\intextsep}{16pt}
\setlength{\belowcaptionskip}{1pt}

%%% Make a bit of additional space for footnotes
%%% No.
%\addtolength{\skip\footins}{4pt}
%\renewcommand{\footnoterule}{%
%   \kern -7pt                   % call this kerna
%   \hrule height 0.4pt width 0.4\columnwidth
%   \kern 6.6pt                  % call this kernb
%}

% Set section and TOC numbering depth to subsection
\setsecnumdepth{subsection}
\settocdepth{subsection}

% Turn off american style paragraph indentation and add some space to be
% printed when a new paragraph starts.

\setlength{\parindent}{0pt}
\addtolength{\parskip}{2pt}

\newcommand{\professor}[1]{\def\@professor{#1}}
\renewcommand{\maketitlehookb}%
{\vspace{2em}\centering\Large\@professor\vspace{0.3\textheight}}

% A bit spacier tabulars and lists
\setlength{\extrarowheight}{4pt}
\setlength{\itemsep}{10pt}
\renewcommand{\arraystretch}{1.2}

% This provides a frontend to set the lecture date into the header
\newcommand{\lecturedate}[1]{\def\@lecdate{#1}}
\makeoddhead{Ruled}{\@lecdate}{}{\normalfont\rightmark}

\makeatother

% This defines how theorems should look. Best leave as is.
\theoremstyle{plain}
\theoremseparator{:\quad}
\theoremprework{}
\theoremindent2em
\theoremheaderfont{\sffamily\bfseries}
\theorembodyfont{\normalfont}
\theoremsymbol{}

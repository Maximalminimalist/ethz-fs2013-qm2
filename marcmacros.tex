% -\_/ % Marc supplement % \_/- %
\newcommand{\td}{\mathrm d }

% TikZ
\newcommand{\qd}{\tikz\draw (0,0) rectangle +(2ex,2ex);\\}
% QM1
\newcommand{\bra}[1]{\left\langle #1\right|}
\newcommand{\ket}[1]{\left| #1\right\rangle}
\newcommand{\braket}[2]{\left\langle #1\right|\left.\! #2\right\rangle}
\newcommand{\ketbra}[2]{\ensuremath{ {\ket{#1} \!\bra{#2}}}}
\newcommand{\proj}[1]{\ensuremath{ {\ket{#1} \!\bra{#1}}}}
\newcommand{\sumproj}[1]{\ensuremath{ {\sum_{#1}\proj{#1}}}}
\newcommand{\sand}[3]{\ensuremath{\left\langle {#1}\vphantom{#3} \right\rvert{#2} \left\lvert{#3}\vphantom{#1}\right\rangle}}


\newcommand{\vevj}[2]{\left\langle {#1} \right\langle_{#2} }
\newcommand{\vev}[1]{\left\langle {#1} \right\rangle }


\newcommand{\comm}[2]{\left[ {#1}, {#2} \right] }
\newcommand{\acomm}[2]{\left\{ {#1}, {#2} \right\} }

\newcommand{\norm}[1]{\left\lvert {#1} \right\rvert }



%calculus
\newcommand{\ddd}[2]{\frac{\mathrm d ^{2} #1}{\mathrm d #2 ^{2}}}
\newcommand{\dd}[2]{\frac{\mathrm d #1}{\mathrm d #2}}
\newcommand{\ddn}[3]{\frac{\mathrm d ^{#1} #2}{\mathrm d #3 ^{#1}}}
\newcommand{\pdd}[2]{\frac{\partial #1}{\partial #2}}
\newcommand{\pddd}[2]{\frac{\partial^{2} #1}{\partial #2 ^{2}}}
\newcommand{\pdddm}[3]{\frac{\partial^{2} #1}{\partial #2 \partial #3}}
\newcommand{\pddn}[3]{\frac{\partial^{#1} #2}{\partial #3 ^{#1}}}
\newcommand{\wrt}[1]{\; d #1}
\newcommand{\de}{differential equation}
\newcommand{\diverg}{\operatorname{{\mathrm div}}}
\newcommand{\curl}{\operatorname{{\mathrm curl}}}
\newcommand{\sgn}{\operatorname{{\mathrm sgn}}}
\newcommand{\Ltran}[1]{\mathcal{L} \left( #1 \right) }
\newcommand{\Ftran}[1]{\mathcal{F} \left( #1 \right) }
\newcommand{\Ftrani}[1]{\mathcal{F}^{-1} \left( #1 \right) }

%vectors + complex
\newcommand{\Arg}{\operatorname{{\mathrm Arg}}}
\newcommand{\Res}[2]{\mathrm{Res} \left( #1 , #2 \right) }
%\DeclareMathAlphabet{\m}{OT1}{cmss}{m}{sl} % for matrices
\newcommand{\vtr}[1]{\boldsymbol{\mathrm{#1}}}
% \DeclareMathAlphabet doesn't do symbols bold so can't be used for 
% \vtr
\newcommand{\Log}{\operatorname{{\mathrm Log}}}
\newcommand{\Ln}{\operatorname{{\mathrm Ln}}}
\newcommand{\dotp}{\cdot}
\newcommand{\Int}{\operatorname{{\mathrm Int}}}
\newcommand{\Ext}{\operatorname{{\mathrm Ext}}}
\newcommand{\Real}{\operatorname{{\mathrm Re}}}
\newcommand{\Imag}{\operatorname{{\mathrm Im}}}
\newcommand{\vwrt}[1]{\dotp {\rm d} \vtr{#1}}
\newcommand{\dist}[2]{d \left( #1 , #2 \right)}

%%% Randnotizen Abstand vom Text
%\marginparsep-0.8cm
%%% alle Randnotizen links statt rechts
%\reversemarginpar

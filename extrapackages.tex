%% OPTIONAL: This is needed to generate an index at the end of the
%% document.
%\usepackage{makeidx}

%% Make document internal hyperlinks whereever possible. (TOC, references)
\usepackage{hyperref}

%% Math related packages

%% The AMS-LaTeX extensions for mathematical typesetting.
\usepackage{amsmath,amssymb,amsfonts,mathrsfs}

%% NTheorem is a reimplementation of the AMS Theorem package. This will allow
%% us to typeset theorems like examples, proofs and similar.
%% NOTE: Must be loaded AFTER amsmath, or the \qed placement will break
\usepackage[thmmarks]{ntheorem}

%% LaTeX' own graphics handling
\usepackage{graphicx}

%% [OPT] To make automatic numbered items
\usepackage{enumerate}

%% [OPT] For the ability to use arithmetic operators for options in LaTeX
\usepackage{calc}
%% [REC] To have bold symbols. Works also with greek letters and symbols.
\usepackage{bm}
%% [REC] To use \bm\mathrm instead of \vec one needs the appropriate greek letters
\usepackage{upgreek}
%%% [OPT] Extension of bigger braces
\usepackage{yhmath}

%% [REC] To have SI-units how they should look like
\usepackage{siunitx}
%% [OPT] To draw sketches
\usepackage{tikz,pgflibraryshapes}
%% And it's recommended libraries which are mostly used.
\usetikzlibrary{positioning,calc,matrix,chains,scopes,fit,patterns,decorations,decorations.pathmorphing,decorations.pathreplacing,arrows,through,backgrounds}
%% [OPT] To have precise 3d-sketches
\usepackage{tikz-3dplot}
%% [OPT] Fancy physics macros (Unfortunately NOT COMPATIBLE with breqn.)
%% \usepackage{physics}

%% [OPT] Book-like tabulars
\usepackage{booktabs}



%% See the TeXed file for more explanations

%% [OPT] Multi-rowed cells in tabulars
%\usepackage{multirow}

%% [REC] Intelligent cross reference package. This allows for nice
%% combined references that include the reference and a hint to where
%% to look for it.
\usepackage{varioref}

%% [OPT] Easily changeable quotes with \enquote{Text}
%\usepackage[german=swiss]{csquotes}

%% [REC] Format dates and time depending on locale
\usepackage{datetime}

%% [OPT] Provides a \cancel{} command to stroke through mathematics.
\usepackage{cancel}

%% [ADV] This allows for additional typesetting tools in
%% mathmode. \mathclap{} will allow you to take away the width of a mathematical statement seen by LaTeX. Basically this packages adds features that lack in the AMSmath package.
\usepackage{mathtools}

%% [OPT] Manual large braces or other delimiters.
%\usepackage{bigdelim, bigstrut}

%% [REC] Alternate vector arrows. Use the command \vv{} to get scaled
%% vector arrows.
\usepackage[h]{esvect}

%% [NEED] Some extensions to tabulars and array environments.
\usepackage{array}

%% [OPT] Provides \unit[1]{N} as a means to facilitate unit
%% typesetting as well as \nicefrac{}{} command that prints a fraction
%% in text-height. Very useful for fractions inside matrices.
%\usepackage{units}

%% [NEED] Allows to rotate elements
\usepackage{rotating}

%% [OPT] LaTeX epic/eepic graphics format support.
%\usepackage{epic,eepic}

%% [OPT] Postscript support via pstricks graphics package. Very
%% diverse applications.
%\usepackage{pstricks,pst-all}

%% [?] This seems to allow us to define some additional counters.
%\usepackage{etex}

%[OPT] To write 1\textsuperscript as \nth{1} etc.
\usepackage{nth}

%% [ADV] XY-Pic to typeset some matrix-style graphics
%\usepackage[all]{xy}
%% [ADV] Automatic line breaking math environments. Very fancy, but in alpha stage. 
%% Only recommended for advanced users.
\usepackage{flexisym}
\usepackage{breqn}

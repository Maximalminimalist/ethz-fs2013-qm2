\chapter{The hydrogen atom}
\section{Basics}
Two body problem proton (1)-electron (2)
\begin{dmath}[]
	H=-\frac{\hbar^2}{2m_2}\vtr{\nabla}_{1}^{2}-\frac{\hbar^2}{2m_2}\nabla_{2}^{2}+V\left( \vtr{r}_1-\vtr{r}_2 \right)
\end{dmath}
new variables
\begin{dgroup*}[]
	\begin{dmath}[]
		R=\frac{m_2r_1+m_2r_2}{m_1+m_2}
	\end{dmath}
	\begin{dmath}[]
		r=r_1-r_2
	\end{dmath}
	\begin{dmath}[]
		M=m_1+m_2
	\end{dmath}
	\begin{dmath}[]
		m=\frac{m_1m_2}{m_1+m_2}
	\end{dmath}
	\begin{dmath}[]
		H=-\frac{\hbar^2}{2M}
	\end{dmath}
\end{dgroup*}

\section{Spin-orbit term}
naive ``derivation'' 
\begin{enumerate}[(1)]
	\item Electron with spin $\to$ magnetic dipole moment
		\begin{dmath}[]
			\vtr{\upmu}=\frac{e}{m}\frac{g}{2}\vtr{s}
		\end{dmath},
		\begin{dseries}[]
			\begin{math}
				\mu=\frac{e}{T}\pi r^2
			\end{math},
			\begin{math}
				s=\frac{2\pi}{T}mr^2
			\end{math},
			\begin{math}
				g\simeq \text{ (from Dirac)}
			\end{math}
		\end{dseries}
	\item Electron feeds magnetic field due to the proton
		\begin{dgroup*}[]
			\begin{dmath}[]
				\vtr{E}\sim \frac{e}{r3}\vtr{r}
			\end{dmath}
			\begin{dmath}[]
				\hiderel{\to}\vtr{B}=-\frac{1}{c^2}\vtr{\nabla}\times \vtr{E}
				=-\frac{1}{mc^2r^3}\vtr{p}\times \vtr{r}=\frac{-\vtr{L}}{mc^2r^3}
			\end{dmath}
		\end{dgroup*}
		wrong by factor $2$ (Thomas precession)
\end{enumerate}
\ldots
correct result
\begin{dmath}[]
	H_{\text{SO}}=\frac{Ze^2}{2mc^2}\frac{1}{r^3}\vtr{L}\cdot\vtr{S}\condition*{\left( \sim -\vtr{\upmu}\cdot \vtr{B} \right)}
\end{dmath}
To describe spin
\begin{dgroup}[]
	\begin{dmath}[]
		\ket{n,\ell,\left( s=\frac{1}{2},m_s \right)}=\psi_{n\ell m_{\ell}m_{s}}=\psi_{n\ell m_{\ell}}\left( r,\theta\varphi \right)\chi_{m_{s}}
	\end{dmath}
	\begin{dsuspend}
		with $\chi_{m_s}$ spin-orbit 
	\end{dsuspend} 
	\begin{dmath}[]
		\begin{pmatrix}
			1\\
			0
		\end{pmatrix}
		\begin{pmatrix}
			0\\
			1
		\end{pmatrix}
	\end{dmath}
\end{dgroup}
Note 
%\begin{dgroup}[]
	%\begin{dmath}[]
	%	\Delta E_{\text{SO}}=0
	%\end{dmath}
	%\begin{dsuspend}
	%	for
	%\end{dsuspend}
	%\begin{dmath}[]
	%	\ell\neq \ell'
	%\end{dmath}
	%\begin{dsuspend}
	%	since
	%\end{dsuspend}
%	\begin{dmath}[]
%		\comm{L^2}{H_{\text{SO}}}=\comm{L^3}{L\cdot S}=0
%	\end{dmath}
%\end{dgroup}
$H_{\text{SO}}$ ``mixes'' states with same $\ell$, but different $m_{\ell},m_{\ell}'$\\
$\to$ use degenerate perturbation theory with $2\cdot \left( 2\ell +1 \right)\times \underbrace{2}_{\text{spin}}\underbrace{\left( 2\ell +1 \right)}_{ m_{\ell}}$ matrix
\begin{dmath}[]
	\ev{n,\ell,m_{\ell}',m_s'}{H_{\text{SO}}}{n,\ell,m_{\ell},m_{s}}\to \text{diagonalize}
\end{dmath}
recall degenerate perturbation theory $\to$ find  ``good'' linear combination that diagonalize this matrix by looking for symmetry use total angular momentum
\begin{dmath}[]
	J\equiv L+S
\end{dmath}
for $\ell=0\quad j=\frac{1}{2}$, for $\ell\neq 0\quad j=\ell\pm \frac{1}{2}$. Use states
\begin{dmath}[]
	\ket{n,\ell,j,m_{j}}
\end{dmath}
\begin{dmath}[]
	\ket{n,\ell,j,m_{j}}=\sum_{m_{\ell},m_{s}}^{}\ket{n,\ell,m_{\ell},m_{s}}\underbrace{\braket{n,\ell,m_{\ell},m_{s}}{n,\ell,j,m_{j}}}_{\text{Clebsch-Gordan}}
\end{dmath}
use 
\begin{dgroup}[]
	\begin{dmath}[]
		J^{2}=L^2+2L\cdot S+S^2
	\end{dmath}
	\begin{dmath}[]
		L\cdot S =\frac{1}{2}\left( J^2-L^2-S^2 \right)
	\end{dmath}
\end{dgroup}
$\ket{n,\ell,j,m_{j}}$ are eigenstates of 
\begin{dmath}[]
	H_0,L^2,S^2,J^2,J_z
\end{dmath}
with eigenvalues
\begin{dmath}[]
	E_n,\hbar^2\ell\left( \ell+1 \right),\hbar^2\frac{3}{4},\hbar^2j\left( j+1 \right),\hbar m_j
\end{dmath}
\begin{dmath}[]
	\Delta E_{\text{SO}}=\ev{n,\ell,j,m_{j}}{H_{\text{SO}}}{n,\ell,j,m_j}
\end{dmath}
for $\ell=0$
\begin{dgroup}[]
	\begin{dmath}[]
		\Delta E_{\text{SO}}=0
	\end{dmath}
	\begin{dsuspend}
		for $\ell \neq 0$
	\end{dsuspend}
	\begin{dmath}[]
		\Delta E_{\text{SO}}=\frac{Ze^2}{2m^2c^2}\ev{n,\ell,j,m_{j}}{\frac{1}{r^3}\frac{1}{2}\left( J^2-L^2-S^2 \right)}{n,\ell,j,m_{j}}
		=\frac{Ze^2}{2m^2c^2}\vev{\frac{1}{r^3}}\frac{\hbar^2}{2}\left( j\left( j+1 \right)-\ell\left( \ell+1 \right)-\frac{3}{4} \right)
		=-E_{n}\frac{\left( Z\alpha \right)^2}{2n\left( \ell+\frac{1}{2} \right)}
		\begin{cases}
			\frac{1}{\ell+1}& j=\ell+\frac{1}{2}\\
			-\frac{1}{\ell}& j=\ell-\frac{1}{2}
		\end{cases}
	\end{dmath}
\end{dgroup}
\section{Darwin term}
Sloppy consideration electron position fluctuates by $\delta r\simeq \lambda_c\simeq \frac{\hbar}{mc}$ electron feels average potential
\begin{dmath}[]
	\vev{V\left( r+\delta r \right)}=\vev{V(r)}+\underbrace{\frac{1}{2}\vev{\vtr{\updelta}}r\cdot \vtr{\nabla}\vtr{\updelta} r\cdot \nabla V}_{}
\end{dmath}
correct result is 
\begin{dmath}[]
	H_{D}=\frac{\hbar^2}{8 m^2c^2}\nabla^2V=\frac{\pi \hbar^2Ze^2}{2m^2c^2}\delta(r)
\end{dmath}
only for $\ell=0$!
\begin{dmath}[]
	\Delta E_D=\ev{n,\ell,j,m_{j}}{H_{D}}{n,\ell,j,m_{j}}
	=\frac{\pi\hbar^2Ze^2}{2m^2c^2}\norm{\psi_{n\ell}(0)}^2
	=-E_{n}\frac{\left( Z\alpha \right)^{2}}{n}\delta_{\ell 0}
\end{dmath}


\section{Periodic perturbations}
Let $V(t)=\left(Ve^{-i\omega t}+V^+e^{i\omega t}\right)$ for $t>t_0=0$\\
The transition probability $P_{i\to f}$ is given by
\begin{dgroup}[]

\begin{dmath}[]
	P_{i\to f}&=\frac{1}{\hbar^2}\left|\int_{t_0}^t\dd{t'}
\end{dmath}

\begin{dmath}[]
	\left(Ve^{-i\omega t'}+V^+e^{i\omega t'}\right)\right|^2
\end{dmath}

\begin{dmath}[]
	=\frac{\pi t}{\hbar^2}\left[|V_{fi}|^2\frac{\sin^2\left(\frac{t}{2}(\omega_{fi}-\omega)\right)}{\pi t\left(\frac{\omega_{fi-\omega}}{2}\right)^2}+|V^+_{fi}|^2\frac{\sin^2\left(\frac{t}{2}(\omega_{fi}+\omega)\right)}{\pi t\left(\frac{\omega_{fi}+\omega}{2}\right)^2}\right. 
\end{dmath}

\begin{dmath}[]
	+\left.\Re(V_{fi}V^+_{fi})\mathcal{F}(\omega_{fi},\omega)\right]
\end{dmath}
\end{dgroup}

Behavior for t large ($t\gg \frac{2\pi}{\omega}$). Recall that ($\sin^2\to \delta$-distribution) interference term vanishes. For the transition rate we have:
\begin{dmath}[]
\Gamma_{i\to f}=\frac{2\pi}{\hbar}\left(|V_{fi}|^2\delta(\underbrace{E_f-E_i-\hbar\omega}_{\stackrel{E_f=E_i+\hbar\omega}{\text{absorbtion } \hbar\omega}})+|V^+_{fi}|^2\delta(\underbrace{E_f-E_i+\hbar\omega}_{\stackrel{E_f=E_i+\hbar\omega}{emmission} \hbar\omega})\right)
\end{dmath}


\section{The interaction picture}
Consider again the evolution operator $\ket{\psi(t)}=U(t,t_0)\ket{\psi(t_0)}$ with the following properties:
\begin{enumerate}
\item $U(t,t_0)$
\item $U(t,t_1)U(t_1,t_0)=U(t,t_0)$
\item 
\begin{dgroup}[]
	\begin{dmath}[]
	i\hbar\Pdiff{}{t}\ket{\psi(t)}&=H(t)\ket{\psi(t)}
	\end{dmath}
	\begin{dmath}[]
	&=i\hbar\Pdiff{}{t}\ket{\psi(t)}=&H(t)U(t,t_0)\ket{\psi(t_0)}
	\end{dmath}
\end{dmath}
\end{dgroup}
\end{enumerate}


such that U satisfies:
\begin{dmath}[]
i\hbar U(t,t_0)=H(t)U(t,t_0)
\end{dmath}
with the formal solution:
\begin{dmath}[]
U(t,t_0)=1+\frac{1}{i\hbar}\int_{t_0}\dd{t'}H(t')U(t',t_0)
\end{dmath}
Now solve by iteration: $U(t,t_0)=\sum_{n=0}^{\infty}=1+U^{(0)}+\cdots$
\begin{dgroup}[]
	\begin{dmath}[]
	U^{(1)}\ket{t,t_0}&=\frac{1}{i\hbar}\int_{t_0}^t\dd{t_1}H(t_1)
	\end{dmath}
	\begin{dmath}[]
	U^{(2)}\ket{t,t_0}&=\frac{1}{i\hbar}\int_{t_0}^t\dd{t_2}H(t_2)\int_{t_0}^{t_2}\dd{t_1}H(t_1)
	\end{dmath}
	\begin{dmath}[]
	=\frac{1}{2!i\hbar}\int_{t_0}^t\dd{t_1}\dd{t_2}\mathcal{T}(H(t_1)H(t_2))
	\end{dmath}
	\end{dmath}
\end{dgroup}

So we arrive at:
\begin{dgroup}[]
	\begin{dmath}[]
	U^{(n)}(t,t_0)&=(i\hbar)^{-n}\int_{t_0}^t\dd{t_n}H(t_n)\int_{t_0}^{t_n}\dd{t_{n-1}}H(t_{n-1})\cdots\int_{t_0}^{t_2}\dd{t_1}H(t_1) 
	\end{dmath}
	\begin{dmath}[]
	=\frac{1}{n!}(i\hbar)^{-n}\int_{t_0}^t\dd{t_1}\int_{t_0}^t\dd{t_{n-1}}\cdots\int_{t_0}^t\dd{t_1}\mathcal{T}(H(t_n)\ldots H(t_1))
	\end{dmath}
\end{dgroup}
where $\mathcal{T}$ is the \textbf{Time-ordering operator}:
\begin{dgroup}[]
	\begin{dmath}[]
	\mathcal{T}(H(t_n)\ldots H(t_1))=H(t_{\tau(1)})\cdots H(t_{\tau(n)})
	\end{dmath}
	
	\begin{dmath}[]
	 t_{\tau(1)}>t_{\tau(2)}>\cdots>t_{\tau(n)}
	\end{dmath}
\textbf{Full solution}
\begin{dmath}[]
U(t,t_0)=\mathcal{T}\left[e^{-\frac{i}{\hbar}\int_{t_0}^tH(t')\dd{t'}}\right]
\end{dmath}


\section{The adiabatic approximation}
Here $V(t)$ is not small but changes \textit{slowly}.\\
ADD PICTURES\\
\newline
\begin{shaded}
\textbf{Theorem}\\
For an \textbf{adiabatic change} $H_i\to H_f$, a system that is initially in the n-th eigenstate of $H_i$ will evolve into into the n-th eigenstate of $H_f$ (no level crossings)
\end{shaded}
\textbf{Proof:}\\
Let $H(t)=E_n(t)\ket{\psi_n(t)}$
\begin{dgroup}[]
	\begin{dmath}[]
	\ket{\psi(t)}&=\sum_{n}c_n(t)\underbrace{e^{(i\hbar)^{-1}\int_0^tE_n(\tau)\dd{\tau}}}_{e^{i\varepsilon_n}}\ket{\psi_n(t)}
	\end{dmath}
	
	\begin{dmath}[]
	=\sum_{n}c_n(t)e^{i\varepsilon_n}\ket{\psi_n(t)}
	\end{dmath}
\end{dgroup}
Plug into Schrödinger equation:
\begin{dmath}[]
i\hbar\sum_{n}\left(\dot{c}_n\ket{\psi_n}+c_n\ket{\dot{\psi}_n}+ic_n\dot{\varepsilon}_n\ket{\psi_n}\right)=\sum_nc_ne^{i\varepsilon_n}H\ket{\psi_n}
\end{dmath}
Multiply by $\bra{\psi_m}$
\begin{dmath}[]
\dot{c}_m=-\varepsilon c_n\underbrace{\braket{\psi_m|\dot{\psi}_n}}_{\text{compute}} e^{i(\varepsilon_n-\varepsilon_m)}
\end{dmath}
Now for $m\neq n$
\begin{dgroup}[]
	\begin{dmath}[]
	H\ket{\psi_n}=E_n\ket{\psi_n}
	\end{dmath}
	
	\begin{dmath}[]
	\dot{H}\ket{\psi_n}+H\ket{\dot{\psi_n}}=\dot{E}_n\ket{\psi_n}+E_n\ket{\dot{\psi}_n}
	\end{dmath}
	
	\begin{dmath}[]
	\mel{\psi_m}{\dot{H}}{\psi_n}=(E_n-E_m)\braket{\psi_m}{\dot{\psi}_n}
	\end{dmath}
\end{dgroup}
Put into equation for $\dot{c}_m$
\begin{dmath}[]
\dot{c}_m=-c_m\braket{\psi_m}{\dot{\psi}_m}-\sum_{n\neq m}c_n\frac{\mel{\psi_m}{\dot{H}}{\psi_n}}{E_n-E_m}e^{i(\varepsilon_n-\varepsilon_m)}
\end{dmath}
where the last sum is an adiabatic approximation for $\dot{H}$ small $\to 0$ and also must not have degeneracy.
\begin{dgroup}[]
	\begin{dmath}[]
	\dot{c}_m(t)=-c_m\braket{\psi_m}{\dot{\psi}_m}
	\end{dmath}
	
	\begin{dmath}[]
	c_m(t)=c_m(t_0)e^{i\gamma_m(t)} \quad \gamma_m=i\int_{t_0}^{t}\dd{\tau}\braket{\psi_m(\tau)}{\dot{\psi}_m(\tau)}
	\end{dmath}
\end{dgroup}
If the system is in state $\psi_m$ at time $t$, $H$ remains in this state
\begin{dgroup}[]
	\begin{dmath}[]
	\psi(t)&=c_m(t)e^{i\varepsilon_m(t)}\ket{\psi_m(t)}
	\end{dmath}
	
	\begin{dmath}[]
	=e^{i\gamma_m(t)}e^{i\varepsilon_m(t)}\ket{\psi_m(t)}
	\end{dmath}
\end{dgroup}
where $e^{i\gamma_m(t)}$ is the geometric phase and $e^{i\varepsilon_m(t)}$ the dynamic phase.
(Berry phase)

\section{Interaction of matter with classical radiation}
Here, radiation is treated as a classical fild
\section{Basics from EM and QM I}
External classical field $\vtr B=\nabla \times \vtr A$, $\vtr E=-\nabla\phi-\dot{\vtr{A}}$\\
(in relativity, $\phi,\vtr A\to A^\mu$)\\

Physics is invariant under gauge tranformations:
\begin{dmath}[]
A^\mu\to A^\mu+\gamma^\mu\chi\begin{cases}
\vtr A\to \vtr A+\nabla\chi(\vtr r, t)\\
\phi \to \phi-\frac{1}{c}\dot{\chi}(\vtr r,t)
\end{cases}
\end{dmath}

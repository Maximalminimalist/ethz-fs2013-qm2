\chapter*{Introduction}
These are my lecture notes of the lecture. You are welcome to tell any mistakes to: mmaetz AT student.ethz.ch. Unfortunately, some lectures are missing because lenovo/IBM is incompetent to give me a working laptop (thinkpad) after two months (even after about 30 e-mails and 10 phone calls).
\chapter{Approximation methods for stationary problems}
\begin{itemize}
	\item standard QM problem:
		\begin{itemize}
			\item given $\ket{\psi(t_0)}$
			\item wanted 
				\begin{dgroup*}[]
					\begin{dmath*}[]
						\ket{\psi(t)}=U\left( t,t_0 \right)\ket{\psi(t_0)}
					\end{dmath*}
					\begin{dmath*}[]
						U=e^{-\frac{i}{\hbar}H\left( t-t_0 \right)}
					\end{dmath*}
				\end{dgroup*}
		\end{itemize}
	\item for time independent $H$
		\begin{dmath}[]
			U=e^{-\frac{i}{\hbar}H\left( t-t_0 \right)}
		\end{dmath}
		find eigenvalue and eigenstates (diagonalize $H$)
\end{itemize}
but: most problems cannot be solved exactly $\to$ find approximate solution.
\section{Time-independent perturbation theory, non-degenerate case}
Assume:
\begin{dmath}[]
	H=H_0+\lambda V
\end{dmath}
with $H_0$ the Hamiltonian that I can solve (``free'' Hamiltonian) and the perturbation $V$ ``small'' and $\lambda$ a dimensionless bookkeeping part.
\begin{dgroup*}[]
	\begin{dmath*}[]
		\lambda\to 0 \condition*{H\to H_0}
	\end{dmath*}
	\begin{dmath*}[]
		\lambda\to 1 \condition*{\text{full } H}
	\end{dmath*}
\end{dgroup*}
We know
\begin{dmath*}[]
	\ket{\psi_{n}^{(0)}}, E_{n}^{(0)}
\end{dmath*}
with
\begin{dgroup*}[]
	\begin{dmath*}[]
		H_0\ket{\psi_{n}^{(0)}}=E_{n}^{(0)}\ket{\psi_{n}^{(0)}}
	\end{dmath*}
	\begin{dsuspend}
		with
	\end{dsuspend}
	\begin{dmath*}[]
		\braket{\psi_{n}^{(0)}}{\psi_{m}^{(0)}}=\delta_{mn}
	\end{dmath*}
\end{dgroup*}
(continuous spectrum also understood.)

We want $\ket{\psi_n}$ and $E_n$ with 
\begin{dmath*}[]
	\left( H_0+\lambda V \right)\ket{\psi_n}=E_n\ket{\psi_n}
\end{dmath*}
let
\begin{dgroup*}[]
	\begin{dmath*}[]
		E_n=E_{n}^{(0)}+\lambda E_{n}^{(1)}+\lambda^2 E_{n}^{(2)}+ \dots
	\end{dmath*}
	\begin{dmath*}[]
		\ket{\psi_n}=\ket{\psi_{n}^{(0)}}+\lambda \ket{\psi_{n}^{(1)}}+\lambda^2\ket{\psi_{n}^{(2)}}+ \dots
	\end{dmath*}
\end{dgroup*}
seems obvious, but assumption. (convergence?)
\begin{dgroup*}[]
	\begin{dmath*}[]
		\left( H_0-E_{n}^{(0)} \right)\ket{\psi_{n}^{(0)}}+\lambda\left( \left( H_0-E_{n}^{(0)} \right)\ket{\psi_{n}^{(1)}} \right)
		+\lambda^2\left( \left( H_0-E_{n}^{(0)} \right)\ket{\psi_{n}^{(2)}}+\left( V-E_{n}^{(1)} \right)\ket{\psi_{n}^{(1)}}-E_{n}^{(2)} \right)
		+ \mathcal{O}\left( \lambda^3 \right)=0
	\end{dmath*}
\end{dgroup*}
with $\mathcal{O}(1)$ ``step 0'', $\mathcal{O}(\lambda)$ ``step 1'', $\mathcal{O}(\lambda^2)$ ``step 2''.
\paragraph{Step 0:} nothing to do
\paragraph{Step 1} multiply by $\bra{\psi_{m}^{(0)}}$
\begin{dgroup*}[]
	\begin{dmath*}[]
	\ev{\psi_{m}^{(0)}}{H_0-E_{n}^{(0)}}{\psi_{n}^{(1)}}+\ev{\psi_{m}^{(0)}}{V-E_{n}^{(1)}}{\psi_{n}^{(0)}}=0
	\end{dmath*}
	\begin{dmath*}[]
		\hiderel{=}\left( E_{m}^{(0)}-E_{n}^{(0)} \right)\braket{\psi_{m}^{(0)}}{\psi_{n}^{(1)}}+\ev{\psi_{m}^{(0)}}{V}{\psi_{n}^{(0)}}-E_{n}^{(1)}\delta_{mn}=0
	\end{dmath*}
	\begin{dsuspend}
		to get $\ket{\psi_{n}^{(1)}}$
	\end{dsuspend}
	\begin{dmath*}[]
		\ket{\psi_{n}^{(1)}}=\sum_{m}^{}\underbrace{\braket{\psi_{m}^{(0)}}{\psi_{n}^{(1)}}}\ket{\psi_{m}^{(0)}}
		=\sum_{m}^{}\frac{\ev{\psi_{m}^{(0)}}{V}{\psi_{n}^{(0)}}}{E_{n}^{(0)}-E_{m}^{(0)}}\ket{\psi_{m}^{(0)}}+\ket{\psi_{n}^{(0)}}\braket{\psi_{n}^{(0)}}{\psi_{n}^{(1)}}
	\end{dmath*}
\end{dgroup*}

\section{The variational principle}
Useful to get good estimate of ground-state energy $E_0$ of complicated systems. Claim
\begin{dmath}[compact]
	E_0\leq \frac{\ev{\psi}{H}{\psi}}{\braket{\psi}{\psi}}=\ev{\psi}{H}{\psi}
\end{dmath}
if $\ket{\psi}$ normalized.
\begin{proof}
	Let
	\begin{dgroup}[]
		\begin{dmath}[]
			\ket{\psi}=\sum_{}^{}c_n\ket{\psi_n}
		\end{dmath},
		\begin{dsuspend}
			with
		\end{dsuspend}
		\begin{dmath}[]
			H\ket{\psi_n}=E_n\ket{\psi_n}
		\end{dmath}
		\begin{dsuspend}
			and
		\end{dsuspend}
		\begin{dmath}[]
			\braket{\psi}{\psi}=1
		\end{dmath}
		\begin{dmath}[]
			\hiderel{\Rightarrow}\sum_{}^{}\norm{c_n}^2=1
		\end{dmath}
		\begin{dsuspend}
			then
		\end{dsuspend}
		\begin{dmath}[compact]
			\ev{\psi}{H}{\psi}=\sum_{m,n}^{}c_{m}^{*}c_n\ev{\psi_m}{H}{\psi_n}
			=\sum_{m,n}^{}c_{m}^{*}c_nE_n\underbrace{\braket{\psi_m}{\psi_n}}_{\delta_{mn}}
			=\sum_{n}^{}\norm{c_n}^2E_n
			\geq E_0\sum_{n}^{}\norm{c_n}^2=E_0
		\end{dmath}
	\end{dgroup}
\end{proof}
\begin{expl}[Harmonic oscillator]
	\begin{dmath}[]
		H=-\frac{\hbar^2}{2m}\dvt{}{x}-\frac{m}{2}\omega^2x^2
	\end{dmath}
	(of course we know $E_0=\frac{\hbar}{2}\omega$). Let
	\begin{dgroup}[]
		\begin{dmath}[]
			\psi(x)=Ae^{-bx^2}
		\end{dmath}
		\begin{dsuspend}
			since
		\end{dsuspend}
		\begin{dmath}[]
			\braket{\psi}{\psi}\stackrel{!}{=}1
			= \int_{}^{}\dd x \,\norm{A}^2 e^{-2bx^2}=\norm{A}^2\sqrt{\frac{\pi}{2b}}
		\end{dmath}
		\begin{dsuspend}
			compute
		\end{dsuspend}
		\begin{dmath}[]
			\ev{\psi}{H}{\psi}
			=\norm{A}^2\int_{}^{}\dd x \, e^{-bx^2}\left( -\frac{\hbar^2}{2m}\dvt{}{x}-\frac{m}{2}\omega^2x^2 \right)e^{-bx^2}
			=\ldots
			= \frac{\hbar^2 b}{2m}+\frac{m\omega^2}{8b}=\ev{\psi}{H}{\psi}\geq E_0
		\end{dmath}
	\end{dgroup}
\end{expl}
Minimize with respect to $b$
\begin{dgroup}[]
	\begin{dmath}[]
		\dv{}{b}\ev{\psi}{H}{\psi}=\frac{\hbar^2}{2m}-\frac{m\omega^2}{8b^2}=0
	\end{dmath}
	\begin{dmath}[]
		b_{\text{min}}=\frac{m\omega}{2\hbar}
	\end{dmath}
	\begin{dmath}[]
		E_0\leq \ev{\psi}{H}{\psi}_{\text{min}}=\frac{\hbar\omega}{2}
	\end{dmath}
\end{dgroup}
in this case we get $E_0$ exactly is a coincidence, since Ansatz$=$true wave function.
\section{WKB approximation, semiclassical approximation}
WKB for Wentzel, Kramers, Brillouin (see QMI Ch. 8.3.)
useful for 1-dim problems with ``smooth'' popential. Schrödinger:
\begin{dgroup}[]
	\begin{dmath}[]
		\left( -\frac{\hbar^2}{2m}\dvt{}{x}+V(x) \right)\psi(x)=E\psi(x)
	\end{dmath}
	\begin{dsuspend}
		if
	\end{dsuspend}
	\begin{dmath}[]
		V(x)\equiv V_0 \text{ const.}
	\end{dmath}
	\begin{dmath}[]
		\psi(x)=e^{\pm \frac{i}{\hbar}\sqrt{2m \left( E-V_0 \right)}x}
	\end{dmath}
	\begin{dsuspend}
		if $V(x)$ is slowly varying. Ansatz
	\end{dsuspend}
	\begin{dmath}[]
		\psi(x)=e^{\frac{i}{\hbar}S(x)}
	\end{dmath}
	\begin{dsuspend}
		Ansatz into Schrödinger:
	\end{dsuspend}
	\begin{dmath}[]
		\frac{-i\hbar}{2m}S''+\frac{1}{2m}\left( S' \right)^2+V(x)-E=0
	\end{dmath}
	\begin{dsuspend}
		equivalent to but more complicated than Schrödinger. Note for
	\end{dsuspend}
	\begin{dmath*}[]
		V(x)\equiv V_0
	\end{dmath*}
	\begin{dmath*}[]
		S=\pm \sqrt{2m\left( E-V_0 \right)}\cdot x
	\end{dmath*}
	\begin{dsuspend}
		and
	\end{dsuspend}
	\begin{dmath*}[]
		S''=0
	\end{dmath*}
	\begin{dsuspend}
		first term $\sim \hbar$ vanishes for 
	\end{dsuspend}
	\begin{dmath*}[]
		V(x)\equiv V_0\condition{(classical limit)}
	\end{dmath*},
\end{dgroup}
Let
\begin{dgroup}[]
	\begin{dmath}[]
		S(x)=S_0(x)+\hbar S_1(x)+\mathcal{O}(\hbar^2)
	\end{dmath}
	\begin{dsuspend}
		plug in into differential equation for $S$
	\end{dsuspend}
	\begin{dmath}[]
		\frac{1}{2m}\left( S_0' \right)^2+V(x)-E=0
	\end{dmath}
	\begin{dmath}[]
		\hiderel{\Rightarrow} S_0'=\pm \sqrt{2m\left( E-V(x) \right)}\equiv \pm p(x)
	\end{dmath}
	\begin{dmath}[]
		S_0'S_1'-\frac{1}{2}S_0''=0
	\end{dmath}
	\begin{dmath}[compact]
		\hiderel{\Rightarrow} S_1'=\frac{i}{2}\frac{S_0''}{S_0'}=\frac{i}{2}\frac{p'(x)}{p(x)}
	\end{dmath}
	\begin{dsuspend}
		solve these differential equation
	\end{dsuspend}
	\begin{dmath}[]
		S_0=\pm \int_{}^{x}\dd x'\, p(x')
	\end{dmath}
	\begin{dmath}[]
		S_1=\frac{i}{2}\ln p(x)
	\end{dmath}
	\begin{dmath}[]
		\hiderel{\Rightarrow} \psi(x)=Ae^{\frac{i}{\hbar}\left( S_0+\hbar _1 \right)}
		=\frac{A_{+}}{\sqrt{p(x)}}e^{\frac{i}{\hbar}\int_{}^{}\dd x'\,p(x')}+\frac{A_{-}}{\sqrt{p(x)}}e^{-\frac{i}{\hbar}\int_{}^{}\dd x'\, p(x')}
	\end{dmath}
\end{dgroup}
